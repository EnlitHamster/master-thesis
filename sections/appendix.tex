\begin{appendices}

\chapter{Example VTK pipeline}
\label{apx:quadric-vtk}
	
\begin{cpp}[label=lst:quadricvtk,caption={Example of Quadric Contour in VTK used to produce Figure~\ref{fig:quadric-render-pipeline}.},aboveskip=20pt]
#include <vtkQuadric.h>
#include <vtkSampleFunction.h>
#include <vtkContourFilter.h>
#include <vtkOutlineFilter.h>
#include <vtkPolyDataMapper.h>
#include <vtkActor.h>
#include <vtkProperty.h>
#include <vtkRenderWindow.h>
#include <vtkRenderer.h>
#include <vtkRenderWindowInteractor.h>
#include <vtkImageData.h>
#include <vtkNew.h>
#include <vtkPolyData.h>

int main(int arc, char* argv)
{
	vtkNew<vtkQuadric> quadric;
	quadric->SetCoefficients(.5,1,.2,0,.1,0,0,.2,0,0);

	vtkNew<vtkSampleFunction> sample;
	sample->SetSampleDimensions(200,200,200);
	sample->SetImplicitFunction(quadric);

	vtkNew<vtkContourFilter> contours;
	contours->SetInputConnection(sample->GetOutputPort());
	contours->GenerateValues(5,0.0,1.2);

	vtkNew<vtkPolyDataMapper> contMapper;
	contMapper->SetInputConnection(contours->GetOutputPort());
	contMapper->SetScalarRange(0.0,1.2);

	vtkNew<vtkActor> contActor;
	contActor->SetMapper(contMapper);

	vtkNew<vtkOutlineFilter> outline;
	outline->SetInputConnection(sample->GetOutputPort());

	vtkNew<vtkPolyDataMapper> outlineMapper;
	outlineMapper->SetInputConnection(outline->GetOutputPort());

	vtkNew<vtkActor> outlineActor;
	outlineActor->SetMapper(outlineMapper);
	outlineActor->GetProperty()->SetColor(0,0,0);

	vtkNew<vtkRenderer> ren;
	vtkNew<vtkRenderWindow> renWin;
	renWin->AddRenderer(ren);

	vtkNew<vtkRenderWindowInteractor> iren;
	iren->SetRenderWindow(renWin);

	ren->AddActor(contActor);
	ren->AddActor(outlineActor);
	ren->SetBackground(1,1,1);

	renWin->Render();
	iren->Start();

	return EXIT_SUCCESS;
}
\end{cpp}
	
\chapter{vtkAdapterUtility code generator}
\label{apx:generate-register}
	
In order to generate at before build a register of all the adapters added to the plugin, a Python script has been created to automate this process. This will generate the necessary support class used within the plugin, i.e. \verb|vtkAdapterUtility|. This class is comprised of a fixed header file and an source file that changes depending on the installed adapters. The Listing~\ref{lst:generateheader} shows the code of such script. An example of generated source file is shown in Listing~\ref{lst:vtkAdapterutilityex}

\begin{python}[label=lst:generateheader,caption={generate-header.py script},aboveskip=20pt]
import os

adapter_utility_h_name = "vtkAdapterUtility.h"
adapter_utility_cpp_name = "vtkAdapterUtility.cpp"
adapter_utility_class_name = "vtkAdapterUtility"
adapter_utility_getter_name = "GetAdapter"
adapter_utility_register_name = "s_adapters"
adapter_utility_register_type = "const std::unordered_map<LPCSTR, vtkAdapter*>"

adapter_utility_h = open(adapter_utility_h_name, "w")
adapter_utility_cpp = open(adapter_utility_cpp_name, "w")



### GENERATING HEADER FILE

adapter_utility_h.write("#pragma once\n")
adapter_utility_h.write("\n")
adapter_utility_h.write("/* This file has been automatically generated through the Python header generation utility\n")
adapter_utility_h.write(" * \n")
adapter_utility_h.write(" * This file contains the necessary information to allow the VtkToUnity plugin to know\n")
adapter_utility_h.write(" * that the adapters exist and it can call them. As such, this should be generated every\n")
adapter_utility_h.write(" * time the plugin is built to avoid losing any adapters in the compilation.\n")
adapter_utility_h.write(" */\n")
adapter_utility_h.write("\n")
adapter_utility_h.write("\n")

adapter_utility_h.write("// Utility includes\n")
adapter_utility_h.write("#include <unordered_map>\n")
adapter_utility_h.write("\n")
adapter_utility_h.write("#define NOMINMAX\n")
adapter_utility_h.write("#include <windows.h>\n")
adapter_utility_h.write("\n")
adapter_utility_h.write("#include \"../Singleton.h\"\n")
adapter_utility_h.write("#include \"../vtkAdapter.h\"\n")
adapter_utility_h.write("\n")


adapter_utility_h.write("\n")
adapter_utility_h.write("\n")
adapter_utility_h.write("// This class is used to register the adapters\n")
adapter_utility_h.write(f"class {adapter_utility_class_name}\n")
adapter_utility_h.write("{\n")

# Generating the class for the utility operations

adapter_utility_h.write("public:\n")

# begin public area

adapter_utility_h.write(f"\tstatic vtkAdapter* {adapter_utility_getter_name}(\n")
adapter_utility_h.write("\t\tLPCSTR vtkAdaptedObject);\n")
adapter_utility_h.write("\n")

# end public area

adapter_utility_h.write("private:\n")

# begin private area

adapter_utility_h.write("\t// Map with all the adapters registered in this folder\n")
adapter_utility_h.write(f"\tstatic {adapter_utility_register_type} {adapter_utility_register_name};\n")

# end private area

adapter_utility_h.write("};\n")



### GENERATING SOURCE FILE

adapter_utility_cpp.write("/* This file has been automatically generated through the Python header generation utility\n")
adapter_utility_cpp.write(" * \n")
adapter_utility_cpp.write(" * This file contains the necessary information to allow the VtkToUnity plugin to know\n")
adapter_utility_cpp.write(" * that the adapters exist and it can call them. As such, this should be generated every\n")
adapter_utility_cpp.write(" * time the plugin is built to avoid losing any adapters in the compilation.\n")
adapter_utility_cpp.write(" */\n")
adapter_utility_cpp.write("\n")
adapter_utility_cpp.write("\n")
adapter_utility_cpp.write(f"#include \"{adapter_utility_h_name}\"\n")
adapter_utility_cpp.write("\n")
adapter_utility_cpp.write("// Adapters' header files found in the folder (.h and .hpp)\n")

# Generating the includes of the adapters' header files

classes = []

for f in os.listdir("."):
    if f != adapter_utility_h_name and ( f.endswith(".h") or f.endswith(".hpp") ):
        adapter_utility_cpp.write(f"#include \"{f}\"\n")
        class_name = os.path.splitext(f)[0]
        classes.append(class_name[:1].upper() + class_name[1:])

adapter_utility_cpp.write("\n")
adapter_utility_cpp.write("\n")

# Creating the adapters' mapping

adapter_utility_cpp.write(f"{adapter_utility_register_type} {adapter_utility_class_name}::{adapter_utility_register_name} =");
adapter_utility_cpp.write("\n{\n")

# begin s_adapters init

for clss in classes:    
    adapter_utility_cpp.write(f"\t{{ Singleton<{clss}>::Instance()->GetAdaptingObject(), Singleton<{clss}>::Instance() }},\n")

# end s_adapters init

adapter_utility_cpp.write("};\n")

adapter_utility_cpp.write("\n")
adapter_utility_cpp.write("\n")

adapter_utility_cpp.write(f"vtkAdapter* {adapter_utility_class_name}::{adapter_utility_getter_name}(\n")
adapter_utility_cpp.write("\tLPCSTR vtkAdaptedObject)\n")
adapter_utility_cpp.write("{\n")

# begin GetAdapter

adapter_utility_cpp.write("\tauto itAdapter = s_adapters.find(vtkAdaptedObject);\n")
adapter_utility_cpp.write("\tif (itAdapter != s_adapters.end())\n")
adapter_utility_cpp.write("\t{\n")
adapter_utility_cpp.write("\t\treturn itAdapter->second;\n")
adapter_utility_cpp.write("\t}\n")
adapter_utility_cpp.write("\telse\n")
adapter_utility_cpp.write("\t{\n")
adapter_utility_cpp.write("\t\treturn NULL;\n")
adapter_utility_cpp.write("\t}\n")

# end GetAdapter

adapter_utility_cpp.write("}\n")
\end{python}
    
\begin{cpp}[label=lst:vtkAdapterutilityex,caption={Example vtkAdapterUtility.cpp},aboveskip=20pt]
/* This file has been automatically generated through the Python header generation utility
 * 
 * This file contains the necessary information to allow the VtkToUnity plugin to know
 * that the adapters exist and it can call them. As such, this should be generated every
 * time the plugin is built to avoid losing any adapters in the compilation.
 */


#include "vtkAdapterUtility.h"

// Adapters' header files found in the folder (.h and .hpp)
#include "vtkConeSourceAdapter.h"


const std::unordered_map<LPCSTR, vtkAdapter*> vtkAdapterUtility::s_adapters ={
	{ Singleton<vtkConeSourceAdapter>::Instance()->GetAdaptingObject(), Singleton<vtkConeSourceAdapter>::Instance() },
};


vtkAdapter* vtkAdapterUtility::GetAdapter(
	LPCSTR vtkAdaptedObject)
{
	auto itAdapter = s_adapters.find(vtkAdaptedObject);
	if (itAdapter != s_adapters.end())
	{
		return itAdapter->second;
	}
	else
	{
		return NULL;
	}
}
\end{cpp}
    
\chapter{Python/C++ Performance tests}
\label{apx:streamtracer-performance-tests}

To evaluate the best approach towards introducing introspection into our project, we developed four tests that determine the performances of Python directly accessing \acrshort{vtk}, Python using introspection to access \acrshort{vtk}, C++ using the embedded Python interpreter to directly access \acrshort{vtk} and C++ using the embedded Python interpreter using introspection to access \acrshort{vtk}, the code for which is shown respectively in Listing~\ref{lst:py-native-vtk}, Listing~\ref{lst:py-intro-vtk}, Listing~\ref{lst:cpp-native-vtk} and Listing~\ref{lst:cpp-intro-vtk}. In both Python and C++ we use helper functions that time the execution of the functions we call. Their implementations are respectively shown in Listings~\ref{lst:py-timed-execution} and Listings~\ref{lst:cpp-timed-execution}. Finally, the introspective C++ code uses functions prefixed with \verb|PyVtk_| which are the same functions presented in our plugin's implementation, adapted not to be run inside a Unity plugin\footnote{The full code is available on GitHub at \url{https://github.com/EnlitHamster/Python-Embedding}.}

\begin{python}[label=lst:py-native-vtk,caption={Native Python VTK benchmark script},aboveskip=20pt]
reader = timed_execution("reader_inst", vtkStructuredGridReader)
timed_execution("reader_setfile", reader.SetFileName, ("density.vtk",))
timed_execution("reader_update", reader.Update)

seeds = timed_execution("seeds_inst", vtkPointSource)
timed_execution("seeds_setradius", seeds.SetRadius, (3.0,))
timed_execution("seeds_setcenter_reader_getoutput_getcenter", seeds.SetCenter, (reader.GetOutput().GetCenter(),))
timed_execution("seeds_setnumberofpoints", seeds.SetNumberOfPoints, (100,))

streamer = timed_execution("streamer_inst", vtkStreamTracer)
timed_execution("streamer_setinputconn_reader_getoutputport", streamer.SetInputConnection, (reader.GetOutputPort(0),))
timed_execution("streamer_setsourceconn_seeds_getoutputport", streamer.SetSourceConnection, (seeds.GetOutputPort(0),))
timed_execution("streamer_setmaxpropagation", streamer.SetMaximumPropagation, (1000,))
timed_execution("streamer_setinitialintegstep", streamer.SetInitialIntegrationStep, (.1,))
timed_execution("streamer_setintegdirboth", streamer.SetIntegrationDirectionToBoth)

outline = timed_execution("outline_inst", vtkStructuredGridOutlineFilter)
timed_execution("outline_setinputconn_reader_getoutputport", outline.SetInputConnection, (reader.GetOutputPort(0),))
\end{python}

\begin{python}[label=lst:py-intro-vtk,caption={Introspective Python VTK benchmark script},aboveskip=20pt]
introspector = timed_execution("introspector_inst", Introspector)

reader = timed_execution("reader_inst", introspector.createVtkObject, ("vtkStructuredGridReader",))
timed_execution("reader_setfile", introspector.setVtkObjectAttribute, (reader, "FileName", "s", "density.vtk"))
timed_execution("reader_update", introspector.updateVtkObject, (reader,))

seeds = timed_execution("seeds_inst", introspector.createVtkObject, ("vtkPointSource",))
timed_execution("seeds_setradius", introspector.setVtkObjectAttribute, (seeds, "Radius", "f", 3.0))
center = timed_execution("reader_getoutputgetcenter", introspector.genericCall, (introspector.vtkInstanceCall(reader, "GetOutput", ()), "GetCenter", ()))
timed_execution("seeds_setcenter_reader_getoutput_getcenter", introspector.setVtkObjectAttribute, (seeds, "Center", "f3", introspector.outputFormat(center)))
timed_execution("seeds_setnumberofpoints", introspector.setVtkObjectAttribute, (seeds, "NumberOfPoints", "d", 100))

streamer = timed_execution("streamer_inst", introspector.createVtkObject, ("vtkStreamTracer",))
timed_execution("streamer_setinputconn_reader_getoutputport", introspector.vtkInstanceCall, (streamer, "SetInputConnection", (introspector.getVtkObjectOutputPort(reader),)))
timed_execution("streamer_setsourceconn_seeds_getoutputport", introspector.vtkInstanceCall, (streamer, "SetSourceConnection", (introspector.getVtkObjectOutputPort(seeds),)))
timed_execution("streamer_setmaxpropagation", introspector.setVtkObjectAttribute, (streamer, "MaximumPropagation", "d", 1000))
timed_execution("streamer_setinitialintegstep", introspector.setVtkObjectAttribute, (streamer, "InitialIntegrationStep", "f", .1))
timed_execution("streamer_setintegdirboth", introspector.vtkInstanceCall, (streamer, "SetIntegrationDirectionToBoth", ()))

outline = timed_execution("outline_inst", introspector.createVtkObject, ("vtkStructuredGridOutlineFilter",))
timed_execution("outline_setinputconn_reader_getoutputport", introspector.vtkInstanceCall, (outline, "SetInputConnection", (introspector.getVtkObjectOutputPort(reader),)))
\end{python}

\begin{cpp}[label=lst:cpp-native-vtk,caption={Native C++ VTK benchmark source code.},aboveskip=20pt]
PyObject *init_python()
{
	/* Activating virtual environment */
	const wchar_t *sPyHome = L"venv";
	Py_SetPythonHome(sPyHome);

	/* Initializing Python environment and setting PYTHONPATH. */
	Py_Initialize();

	/* Both the "." and cwd notations are left in for security, as after being built in
		a DLL they may change. */
	PyRun_SimpleString("import sys\nimport os");
	PyRun_SimpleString("sys.path.append( os.path.dirname(os.getcwd()) )");
	PyRun_SimpleString("sys.path.append(\".\")");

	/* Decode module from its name. Returns error if the name is not decodable. */
	PyObject *pVtkModuleName = PyUnicode_DecodeFSDefault("vtk");
	if (pVtkModuleName == NULL)
	{
		fprintf(stderr, "Fatal error: cannot decode module name\n");
		return NULL;
	}

	/* Imports the module previously decoded. Returns error if the module is not found. */
	PyObject *pVtkModule = PyImport_Import(pVtkModuleName);
	Py_DECREF(pVtkModuleName);
	if (pVtkModule == NULL)
	{
		if (PyErr_Occurred())
		{
			PyErr_Print();
		}
		fprintf(stderr, "Failed to load \"Introspector\"\n");
		return NULL;
	}

	return pVtkModule;
}


PyObject *inst_vtkobj(
	PyObject *pVtkModule,
	LPCSTR classname)
{
	/* Looks for the Introspector class in the module. If it does not find it, returns and error. */
	PyObject* pVtkClass = PyObject_GetAttrString(pVtkModule, classname);
	if (pVtkClass == NULL || !PyCallable_Check(pVtkClass))
	{
		if (PyErr_Occurred())
		{
			PyErr_Print();
		}
		fprintf(stderr, "Cannot find class \"Introspector\"\n");
		if (pVtkClass != NULL)
		{
			Py_DECREF(pVtkClass);
		}
		return NULL;
	}

	/* Instantiates an Introspector object. If the call returns NULL there was an error
		creating the object, and thus it returns error. */
	PyObject *pInstance = PyObject_CallObject(pVtkClass, NULL);
	Py_DECREF(pVtkClass);
	if (pInstance == NULL)
	{
		if (PyErr_Occurred())
		{
			PyErr_Print();
		}
		fprintf(stderr, "Introspector instantiation failed\n");
		return NULL;
	}

	return pInstance;
}


void test_native()
{
	PyObject *pVtkModule = timed_execution<PyObject *>("python_inst", init_python);

	PyObject *pReader = timed_execution<PyObject *>("reader_inst", inst_vtkobj, pVtkModule, "vtkStructuredGrid");
	timed_execution_v("reader_setfile", PyObject_CallMethod, pReader, "SetFileName", "s", "density.vtk");
	timed_execution_v("reader_update", PyObject_CallMethod, pReader, "Update", "");

	PyObject *pSeeds = timed_execution<PyObject *>("seeds_inst", inst_vtkobj, pVtkModule, "vtkPointSource");
	timed_execution_v("seeds_setradius", PyObject_CallMethod, pSeeds, "SetRadius", "d", 3.0);
	PyObject *pOutput = timed_execution<PyObject *>("reader_getoutput", PyObject_CallMethod, pReader, "GetOutput", "");
	PyObject *pCenter = timed_execution<PyObject *>("output_getcenter", PyObject_CallMethod, pOutput, "GetCenter", "");
	timed_execution_v("seeds_setcenter", PyObject_CallMethod, pSeeds, "SetCenter", "(d,d,d)", pCenter);
	timed_execution_v("seeds_setnumberofpoints", PyObject_CallMethod, pSeeds, "SetNumberOfPoints", "i", 100);

	PyObject *pStreamer = timed_execution<PyObject *>("streamer_inst", inst_vtkobj, pVtkModule, "vtkStreamTracer");
	PyObject *pReaderPort = timed_execution<PyObject *>("reader_getoutputport", PyObject_CallMethod, pReader, "GetOutputPort", "i", 0);
	PyObject *pSeedsPort = timed_execution<PyObject *>("seeds_getoutputport", PyObject_CallMethod, pSeeds, "GetOutputPort", "i", 0);
	timed_execution_v("streamer_setinputconn", PyObject_CallMethod, pStreamer, "SetInputConnection", "O", pReaderPort);
	timed_execution_v("streamer_setsourceconn", PyObject_CallMethod, pStreamer, "SetSourceConnection", "O", pSeedsPort);
	timed_execution_v("streamer_setmaxpropagation", PyObject_CallMethod, pStreamer, "SetMaximumPropagation", "i", 1000);
	timed_execution_v("streamer_setinitialintegstep", PyObject_CallMethod, pStreamer, "SetInitialIntegrationStep", "d", 0.1);
	timed_execution_v("streamer_setintegdirboth", PyObject_CallMethod, pStreamer, "SetIntegrationDirectionToBoth", "");

	PyObject *pOutline = timed_execution<PyObject *>("outline_inst", inst_vtkobj, pVtkModule, "vtkStructuredGridOutlineFilter");
	timed_execution_v("outline_setinputconn", PyObject_CallMethod, pOutline, "SetInputConnection", "O", pReaderPort);

	Py_XDECREF(pVtkModule);
	Py_XDECREF(pReader);
	Py_XDECREF(pSeeds);
	Py_XDECREF(pStreamer);
	Py_XDECREF(pOutline);

	timed_execution_v("python_fin", Py_Finalize);
}
\end{cpp}

\begin{cpp}[label=lst:cpp-intro-vtk,caption={Introspective C++ VTK benchmark source code.},aboveskip=20pt]
void test_introspection()
{

	PyObject *pIntrospector = timed_execution<PyObject *>("interpreter_init", PyVtk_InitIntrospector);

	vtkObjectBase
		*pReader = timed_execution<vtkObjectBase *>("reader_inst", PyVtk_CreateVtkObject, pIntrospector, "vtkStructuredGridReader"),
		*pSeeds = timed_execution<vtkObjectBase *>("seeds_inst", PyVtk_CreateVtkObject, pIntrospector, "vtkPointSource"),
		*pStreamer = timed_execution<vtkObjectBase *>("streamer_inst", PyVtk_CreateVtkObject, pIntrospector, "vtkStreamTracer"),
		*pOutline = timed_execution<vtkObjectBase *>("outline_inst", PyVtk_CreateVtkObject, pIntrospector, "vtkStructuredGridOutlineFilter");

	timed_execution_v("reader_setfile", PyVtk_SetVtkObjectProperty, pIntrospector, pReader, "FileName", "s", "density.vtk");

	timed_execution_v("reader_update", PyVtk_ObjectMethod, // pReader->Update()
		pIntrospector,
		pReader,
		"Update",
		"",
		std::vector<vtkObjectBase *>(),
		std::vector<LPCSTR>());


	PyObject *pOutput = timed_execution<PyObject *>("reader_getoutput", PyVtk_ObjectMethod, // pReader->GetOutput()
		pIntrospector,
		pReader,
		"GetOutput",
		"",
		std::vector<vtkObjectBase *>(),
		std::vector<LPCSTR>());

	LPCSTR center = timed_execution<LPCSTR>("reader_getoutput_getcenter", PyVtk_PipedObjectMethodAsString, // pReader->GetOutput()->GetCenter()
		pIntrospector,
		pReader,
		std::vector<LPCSTR>({ "GetOutput", "GetCenter" }),
		std::vector<LPCSTR>({ "", "" }),
		std::vector<vtkObjectBase *>(),
		std::vector<LPCSTR>());

	timed_execution_v("seeds_setradius", PyVtk_SetVtkObjectProperty, pIntrospector, pSeeds, "Radius", "f", "3.0");
	timed_execution_v("seeds_setcenter", PyVtk_SetVtkObjectProperty, pIntrospector, pSeeds, "Center", "f3", center);
	timed_execution_v("seeds_setnumberofpoints", PyVtk_SetVtkObjectProperty, pIntrospector, pSeeds, "NumberOfPoints", "d", "100");

	vtkAlgorithmOutput *pSeedsPort = timed_execution<vtkAlgorithmOutput *>("seeds_getoutputport", PyVtk_GetOutputPort, pIntrospector, pSeeds);

	timed_execution_v("streamer_setinputconn_reader", PyVtk_ConnectVtkObject, pIntrospector, pReader, (vtkAlgorithm *)pStreamer);
	timed_execution_v("streamer_setsourceconn_seeds", PyVtk_ObjectMethod, // pStreamer->SetSourceConnection(pSeeds->GetOutputPort(0))
		pIntrospector,
		pStreamer,
		"SetSourceConnection",
		"o",
		std::vector<vtkObjectBase *>({ pSeedsPort }),
		std::vector<LPCSTR>());

	timed_execution_v("streamer_setmaxpropagation", PyVtk_SetVtkObjectProperty, pIntrospector, pStreamer, "MaximumPropagation", "d", "100");
	timed_execution_v("streamer_setinitialintegstep", PyVtk_SetVtkObjectProperty, pIntrospector, pStreamer, "InitialIntegrationStep", "f", "0.1");
	timed_execution_v("streamer_setintegdirboth", PyVtk_ObjectMethod, // pStreamer->SetIntegrationDirectionToBoth()
		pIntrospector,
		pStreamer,
		"SetIntegrationDirectionToBoth",
		"",
		std::vector<vtkObjectBase *>(),
		std::vector<LPCSTR>());

	timed_execution_v("outline_setinputconn", PyVtk_ConnectVtkObject, pIntrospector, pReader, (vtkAlgorithm *)pOutline);

	timed_execution_v("interpreter_fin", PyVtk_FinalizeIntrospector, pIntrospector);
}
\end{cpp}

\begin{python}[label=lst:py-timed-execution,caption={timed\_execution Python function.},aboveskip=20pt]
time_execution_data = {}
def timed_execution(name, func, args = ()):
	t_start = perf_counter()
	r = func(*args)
	time_execution_data[name] = perf_counter() - t_start
	return r
\end{python}

\begin{cpp}[label=lst:cpp-timed-execution,caption={timed\_execution Cpp functions.},aboveskip=20pt]
typedef std::chrono::high_resolution_clock::time_point time_var;

#define DURATION(a) std::chrono::duration_cast<std::chrono::nanoseconds>(a).count()
#define TIME_NOW() std::chrono::high_resolution_clock::now()

std::unordered_map<LPCSTR, double> time_execution_data;

template<typename R, typename F, typename... A>
R timed_execution(LPCSTR name, F function, A&& ...argv)
{
	time_var start = TIME_NOW();
	R r = function(std::forward<A>(argv)...);
	time_execution_data.insert(std::make_pair(name, DURATION(TIME_NOW() - start) / 1000000000.0f));
	return r;
}

template<typename F, typename... A>
void timed_execution_v(LPCSTR name, F function, A&& ...argv)
{
	time_var start = TIME_NOW();
	function(std::forward<A>(argv)...);
	time_execution_data.insert(std::make_pair(name, DURATION(TIME_NOW() - start) / 1000000000.0f));
}	
\end{cpp}

\chapter{Interfaces' documentation}
\label{apx:api}

In the following pages, we present the API of different parts of the library, in particular Table~\ref{tab:plugin-api} describes the API the plugin exposes to Unity, Table~\ref{tab:introspective-api} the API the introspector exposes to the Infrastructure layer.

\begin{landscape}
	\begin{longtable}[c]{
			>{\raggedright\arraybackslash}p{0.4\linewidth}
			>{\raggedright\arraybackslash}p{0.5\linewidth}
		}
		\multicolumn{1}{c}{\textbf{Function signature}} &
		\multicolumn{1}{c}{\textbf{Explenation}} 
		\\ \hline
	\endhead
	%
	\begin{codify}
	int VtkResource_CallObject(
		LPCSTR classname)
	\end{codify} &
	Instantiates an object of type classname, adds it to the object registry at both Introspection and Plugin level and returns the id in the plugin registry.
	\\
	\begin{codify}
	int VtkResource_CallObjectAndShow(
		LPCSTR classname,
		Float4 color,
		bool wireframe)
	\end{codify} &
	Same as above, but also wraps it in an actor with specified color and whether to show the wireframe.
	\\
	\begin{codify}
	LPCSTR VtkResource_GetAttrAsString(
		const int rid,
		LPCSTR propertyName)
	\end{codify} &
	Retrieves the value of the specificed attribute and returns it as a string.
	\\
	\begin{codify}
	void VtkResource_SetAttrFromString(
		const int rid,
		LPCSTR propertyName,
		LPCSTR format,
		LPCSTR newValue)
	\end{codify} &
	Sets a value from a string. The format identifies how the value has to be parsed.
	\\
	\begin{codify}
	LPCSTR VtkResource_GetDescriptor(
		const int rid)
	\end{codify} &
	Returns a string description of the object, in particular its attributes' name and type, comma separated.
	\\
	\begin{codify}
	LPCSTR VtkResource_CallMethodAsString(
		const int rid,
		LPCSTR method,
		LPCSTR format,
		const char *const *argv)
	\end{codify} &
	Calls a method on a given object and returns its return value as a string.
	\\
	\begin{codify}
	int VtkResource_CallMethodAsVtkObject(
		const int rid,
		LPCSTR method,
		LPCSTR format,
		LPCSTR classname,
		const char *const *argv)
	\end{codify} &
	Calls a method on a given object and returns its return value as a VTK object pointer.
	\\
	\begin{codify}
	void VtkResource_CallMethodAsVoid(
		const int rid,
		LPCSTR method,
		LPCSTR format,
		const char *const *argv)
	\end{codify} &
	Calls a method on a given object and ignores the return value.
	\\
	\begin{codify}
	LPCSTR VtkResource_CallMethodPipedAsString(
		const int rid,
		const int methodc,
		const int formatc,
		const char *const *methodv,
		const char *const *formatv,
		const char *const *argv)
	\end{codify} &
	Calls a methods pipeline starting from a given object and returns its final reutrn value as a string.
	\\
	\begin{codify}
	int VtkResource_CallMethodPipedAsVtkobject(
		const int rid,
		const int methodc,
		const int formatc,
		LPCSTR classname,
		const char *const *methodv,
		const char *const *formatv,
		const char *const *argv)
	\end{codify} &
	Calls a methods pipeline starting from a given object and returns its final reutrn value as a VTK object pointer.
	\\
	\begin{codify}
	void VtkResource_CallMethodPipedAsVoid(
		const int rid,
		const int methodc,
		const int formatc,
		const char *const *methodv,
		const char *const *formatv,
		const char *const *argv)
	\end{codify} &
	Calls a methods pipeline starting from a given object and ignores its final return value.
	\\
	\begin{codify}
	void VtkReousrce_DeleteObject(
		const int rid)
	\end{codify} &
	Deletes an object and its wrappers, freeing its memory space.
	\\
	\begin{codify}
	void VtkResource_Connect(
		LPCSTR connectionType,
		const int sourceRid,
		const int targetRid)
	\end{codify} &
	Connects the sources output to the targets input connection. The connection type identifies the method to be called.
	\\
	\begin{codify}
	void VtkResource_AddActor(
		const int rid,
		Float4 color,
		bool wireframe)
	\end{codify} &
	Adds mapper and actor to the source and adds it to the renderer.
	\\
	\begin{codify}
	bool VtkError_Occurred()
	\end{codify} &
	Checks whether an error occurred on the last call of VtkResource.
	\\
	\begin{codify}
	LPCSTR VtkError_Get()
	\end{codify} &
	Returns the last error occurred on VtkResource. It should always be called, in conjunction with \verb|VtkError_Occurred()|, after any VtkResource call.
	\\
	\caption{Plugin API exposed to Unity.}
	\label{tab:plugin-api}\\
	\end{longtable}
\end{landscape}

\begin{landscape}
	\begin{longtable}[c]{
			>{\raggedright\arraybackslash}p{0.4\linewidth}
			>{\raggedright\arraybackslash}p{0.5\linewidth}
		}
		\multicolumn{1}{c}{\textbf{Function signature}} &
		\multicolumn{1}{c}{\textbf{Explenation}} 
		\\ \hline
	\endhead
	%
	\begin{codify}
	vtkObjectBase *CreateObject(
		LPCSTR classname)
	\end{codify} &
	Instantiates an object of class classname and adds it to the mapping from VTK object to Python object.
	\\
	\begin{codify}
	LPCSTR GetProperty(
		vtkObjectBase *pVtkObject,
		LPCSTR propertyName)
	\end{codify} &
	Retrieves the value of the specificed attribute and returns it as a string.
	\\
	\begin{codify}
	void SetProperty(
		vtkObjectBase *pVtkObject,
		LPCSTR propertyName,
		LPCSTR format,
		LPCSTR newValue)
	\end{codify} &
	Sets a value from a string. The format identifies how the value has to be parsed.
	\\
	\begin{codify}
	LPCSTR GetDescriptor(
		vtkObjectBase *pVtkObject)
	\end{codify} &
	Returns a string description of the object, in particular its attributes' name and type, comma separated.
	\\
	\begin{codify}
	LPCSTR CallMethod_AsString(
		vtkObjectBase *pVtkObject,
		LPCSTR method,
		LPCSTR format,
		vector<vtkObjectBase *> refv,
		vector<LPCSTR> argv)
	\end{codify} &
	Calls a method on a given object and returns its return value as a string.
	\\
	\begin{codify}
	vtkObjectBase *CallMethod_AsVtkObject(
		vtkObjectBase *pVtkObject,
		LPCSTR method,
		LPCSTR format,
		LPCSTR classname,
		vector<vtkObjectBase *> refv,
		vector<LPCSTR> argv)
	\end{codify} &
	Calls a method on a given object and returns its return value as a VTK object pointer.
	\\
	\begin{codify}
	void CallMethod_AsVoid(
		vtkObjectBase *pVtkObject,
		LPCSTR method,
		LPCSTR format,
		vector<vtkObjectBase *> refv,
		vector<LPCSTR> argv)
	\end{codify} &
	Calls a method on a given object and ignores the return value.
	\\
	\begin{codify}
	LPCSTR CallMethodPiped_AsString(
		vtkObjectBase *pVtkObject,
		vector<LPCSTR> methods,
		vector<LPCSTR> formats,
		vector<vtkObjectBase *> refv
		vector<LPCSTR> argv)
	\end{codify} &
	Calls a methods pipeline starting from a given object and returns its final reutrn value as a string.
	\\
	\begin{codify}
	vtkObjectBase *CallMethodPiped_AsVtkObject(
		vtkObjectBase *pVtkObject,
		vector<LPCSTR> methods,
		vector<LPCSTR> formats,
		LPCSTR classname,
		vector<vtkObjectBase *> refv,
		vector<LPCSTR> argv)
	\end{codify} &
	Calls a methods pipeline starting from a given object and returns its final reutrn value as a VTK object pointer.
	\\
	\begin{codify}
	void CallMethodPiped_AsVoid(
		vtkObjectBase *pVtkObject,
		vector<LPCSTR> methods,
		vector<LPCSTR> formats,
		LPCSTR classname,
		vector<vtkObjectBase *> refv,
		vector<LPCSTR> argv)
	\end{codify} &
	Calls a methods pipeline starting from a given object and ignores its final return value.
	\\
	\begin{codify}
	void DeleteObject(
		vtkObjectBase *pVtkObject)
	\end{codify} &
	Deletes an object and its wrappers, freeing its memory space.
	\\
	\begin{codify}
	bool ErrorOccurred()
	\end{codify} &
	Checks whether an error was registered in the error buffer.
	\\
	\begin{codify}
	LPCSTR ErrorGet()
	\end{codify} &
	Returns the last error occurred on an introspection call.
	\\
	\caption{Introspective component's API.}
	\label{tab:introspective-api}\\
	\end{longtable}
\end{landscape}

\chapter{Unity test scenes}
\label{apx:unity-test-scenes}

This chapter presents the relevant code used to create the scenes used to validate the presented plugin.

\begin{figure}[ht!]
    \centering
    \begin{cs}[label=lst:ConeTestVtkInst,caption={C\# calls to create a VTK cone source in Unity (Snippet a) and to update it through a co-routine (Snippet b)}]
/* SNIPPET A */
// Create a backend VTK object through the Introspector.
int id = VtkToUnityPlugin.VtkResource_CallObject("vtkConeSource");

// Set its attributes.
VtkUnityWorkbenchPlugin.SetProperty(id, "Height", "f", 0.1f.ToString());
VtkUnityWorkbenchPlugin.SetProperty(id, "Radius", "f", 0.1f.ToString());
VtkUnityWorkbenchPlugin.SetProperty(id, "Resolution", "d", 200.ToString());

// Directly connect the source to an actor in the backend.
VtkToUnityPlugin.VtkResource_AddActor(id, colour, false);

/* SNIPPET B */
// Defines the parameters of the Cone source for updating
struct Preset
{
    public float Height;
    public float Radius;
    public int Resolution;
}

// Determines which preset to use
private bool useDefault = true;
private Preset defaultPreset = new Preset();
private Preset transformedPreset = new Preset();

// Co-routine to be run alongside the updates to simulate real-time editing
IEnumerator TransformCone()
{
    for(;;)
    {
        // Changes all cones on the scene and yield again in 2 seconds
        foreach (var idPosition in _shapeIdPositions)
        {
            Preset p = useDefault ? defaultPreset : transformedPreset;
            int id = idPosition.Id;
            VtkUnityWorkbenchPlugin.SetProperty(id, "Height", "f", p.Height.ToString());
            VtkUnityWorkbenchPlugin.SetProperty(id, "Radius", "f", p.Radius.ToString());
            VtkUnityWorkbenchPlugin.SetProperty(id, "Resolution", "d", p.Resolution.ToString());

        }
        useDefault = !useDefault;
        yield return new WaitForSeconds(2.0f);
    }
}
    \end{cs}
\end{figure}

\begin{figure}[ht!]
    \centering
    \begin{cs}[label=lst:StremTracerVtkInst,caption={C\# calls to create a VTK stream tracer in Unity.}]
// Creating the VTK resources
int reader = VtkToUnityPlugin.VtkResource_CallObject("vtkStructuredGridReader");
int seeds = VtkToUnityPlugin.VtkResource_CallObject("vtkPointSource");
int streamer = VtkToUnityPlugin.VtkResource_CallObject("vtkStreamTracer");
int outline = VtkToUnityPlugin.VtkResource_CallObject("vtkStructuredGridOutlineFilter");

VtkToUnityPlugin.VtkResource_SetAttrFromString(reader, "FileName", "s", "Data/density.vtk");
VtkToUnityPlugin.VtkResource_CallMethodAsVoid(reader, "Update", "", IntPtr.Zero);

string center = VtkToUnityPlugin.VtkResource_CallMethodPipedAsString(
    reader,
    2,
    2,
    VtkUnityWorkbenchHelpers.MarshalStringArray(new string[] { "GetOutput", "GetCenter" }),
    VtkUnityWorkbenchHelpers.MarshalStringArray(new string[] { "", "" }),
    IntPtr.Zero);

VtkToUnityPlugin.VtkResource_SetAttrFromString(seeds, "Radius", "f", 3.0f.ToString());
VtkToUnityPlugin.VtkResource_SetAttrFromString(seeds, "Center", "f3", center);
VtkToUnityPlugin.VtkResource_SetAttrFromString(seeds, "NumberOfPoints", "d", nPoints.ToString());

VtkToUnityPlugin.VtkResource_Connect("Input", reader, streamer);
VtkToUnityPlugin.VtkResource_Connect("Source", seeds, streamer);
VtkToUnityPlugin.VtkResource_SetAttrFromString(streamer, "MaximumPropagation", "d", 100.ToString());
VtkToUnityPlugin.VtkResource_SetAttrFromString(streamer, "InitialIntegrationStep", "f", 0.1f.ToString());
VtkToUnityPlugin.VtkResource_CallMethodAsVoid(streamer, "SetIntegrationDirectionToBoth", "", IntPtr.Zero);

VtkToUnityPlugin.VtkResource_Connect("Input", reader, outline);

VtkToUnityPlugin.VtkResource_AddActor(streamer, colour, false);
VtkToUnityPlugin.VtkResource_AddActor(outline, colour, false);
    \end{cs}
\end{figure}

\end{appendices}
